% Created 2024-10-20 Sun 16:32
% Intended LaTeX compiler: pdflatex
\documentclass[11pt]{article}
\usepackage[utf8]{inputenc}
\usepackage[T1]{fontenc}
\usepackage{graphicx}
\usepackage{longtable}
\usepackage{wrapfig}
\usepackage{rotating}
\usepackage[normalem]{ulem}
\usepackage{amsmath}
\usepackage{amssymb}
\usepackage{capt-of}
\usepackage{hyperref}
\usepackage[polish]{babel}
\usepackage[T1]{fontenc}
\usepackage[utf8]{inputenc}
\selectlanguage{polish}
\usepackage{caption}
\usepackage{booktabs}
\captionsetup{labelfont=bf}
\usepackage{float}
\author{Piotr Karamon}
\date{21.10.2024r.}
\title{Laboratorium 2 - Teoria współbieżności}
\hypersetup{
 pdfauthor={Piotr Karamon},
 pdftitle={Laboratorium 2 - Teoria współbieżności},
 pdfkeywords={},
 pdfsubject={},
 pdfcreator={Emacs 29.2 (Org mode 9.7.11)}, 
 pdflang={Polish}}

% Setup for code blocks [1/2]

\usepackage{fvextra}

\fvset{%
  commandchars=\\\{\},
  highlightcolor=white!95!black!80!blue,
  breaklines=true,
  breaksymbol=\color{white!60!black}\tiny\ensuremath{\hookrightarrow}}

% Make line numbers smaller and grey.
\renewcommand\theFancyVerbLine{\footnotesize\color{black!40!white}\arabic{FancyVerbLine}}

\usepackage{xcolor}

% In case engrave-faces-latex-gen-preamble has not been run.
\providecolor{EfD}{HTML}{f7f7f7}
\providecolor{EFD}{HTML}{28292e}

% Define a Code environment to prettily wrap the fontified code.
\usepackage[breakable,xparse]{tcolorbox}
\DeclareTColorBox[]{Code}{o}%
{colback=EfD!98!EFD, colframe=EfD!95!EFD,
  fontupper=\footnotesize\setlength{\fboxsep}{0pt},
  colupper=EFD,
  IfNoValueTF={#1}%
  {boxsep=2pt, arc=2.5pt, outer arc=2.5pt,
    boxrule=0.5pt, left=2pt}%
  {boxsep=2.5pt, arc=0pt, outer arc=0pt,
    boxrule=0pt, leftrule=1.5pt, left=0.5pt},
  right=2pt, top=1pt, bottom=0.5pt,
  breakable}

% Support listings with captions
\usepackage{float}
\floatstyle{plain}
\newfloat{listing}{htbp}{lst}
\newcommand{\listingsname}{Listing}
\floatname{listing}{\listingsname}
\newcommand{\listoflistingsname}{List of Listings}
\providecommand{\listoflistings}{\listof{listing}{\listoflistingsname}}


% Setup for code blocks [2/2]: syntax highlighting colors

\newcommand\efstrut{\vrule height 2.1ex depth 0.8ex width 0pt}
\definecolor{EFD}{HTML}{000000}
\definecolor{EfD}{HTML}{ffffff}
\newcommand{\EFD}[1]{\textcolor{EFD}{#1}} % default
\newcommand{\EFvp}[1]{#1} % variable-pitch
\definecolor{EFh}{HTML}{595959}
\newcommand{\EFh}[1]{\textcolor{EFh}{#1}} % shadow
\definecolor{EFsc}{HTML}{005f5f}
\newcommand{\EFsc}[1]{\textcolor{EFsc}{\textbf{#1}}} % success
\definecolor{EFw}{HTML}{884900}
\newcommand{\EFw}[1]{\textcolor{EFw}{\textbf{#1}}} % warning
\definecolor{EFe}{HTML}{a60000}
\newcommand{\EFe}[1]{\textcolor{EFe}{\textbf{#1}}} % error
\definecolor{EFl}{HTML}{3548cf}
\newcommand{\EFl}[1]{\textcolor{EFl}{#1}} % link
\definecolor{EFlv}{HTML}{721045}
\newcommand{\EFlv}[1]{\textcolor{EFlv}{#1}} % link-visited
\definecolor{Efhi}{HTML}{b2e4dc}
\newcommand{\EFhi}[1]{\colorbox{Efhi}{\efstrut{}#1}} % highlight
\definecolor{EFc}{HTML}{595959}
\newcommand{\EFc}[1]{\textcolor{EFc}{\textit{#1}}} % font-lock-comment-face
\definecolor{EFcd}{HTML}{595959}
\newcommand{\EFcd}[1]{\textcolor{EFcd}{\textit{#1}}} % font-lock-comment-delimiter-face
\definecolor{EFs}{HTML}{3548cf}
\newcommand{\EFs}[1]{\textcolor{EFs}{#1}} % font-lock-string-face
\definecolor{EFd}{HTML}{2a5045}
\newcommand{\EFd}[1]{\textcolor{EFd}{\textit{#1}}} % font-lock-doc-face
\definecolor{EFm}{HTML}{7c318f}
\newcommand{\EFm}[1]{\textcolor{EFm}{\textit{#1}}} % font-lock-doc-markup-face
\definecolor{EFk}{HTML}{531ab6}
\newcommand{\EFk}[1]{\textcolor{EFk}{#1}} % font-lock-keyword-face
\definecolor{EFb}{HTML}{8f0075}
\newcommand{\EFb}[1]{\textcolor{EFb}{#1}} % font-lock-builtin-face
\definecolor{EFf}{HTML}{721045}
\newcommand{\EFf}[1]{\textcolor{EFf}{#1}} % font-lock-function-name-face
\definecolor{EFv}{HTML}{005e8b}
\newcommand{\EFv}[1]{\textcolor{EFv}{#1}} % font-lock-variable-name-face
\definecolor{EFt}{HTML}{005f5f}
\newcommand{\EFt}[1]{\textcolor{EFt}{#1}} % font-lock-type-face
\definecolor{EFo}{HTML}{0000b0}
\newcommand{\EFo}[1]{\textcolor{EFo}{#1}} % font-lock-constant-face
\definecolor{EFwr}{HTML}{884900}
\newcommand{\EFwr}[1]{\textcolor{EFwr}{#1}} % font-lock-warning-face
\definecolor{EFnc}{HTML}{a60000}
\newcommand{\EFnc}[1]{\textcolor{EFnc}{\textbf{#1}}} % font-lock-negation-char-face
\definecolor{EFpp}{HTML}{a0132f}
\newcommand{\EFpp}[1]{\textcolor{EFpp}{#1}} % font-lock-preprocessor-face
\definecolor{EFrc}{HTML}{00663f}
\newcommand{\EFrc}[1]{\textcolor{EFrc}{#1}} % font-lock-regexp-grouping-construct
\definecolor{EFrb}{HTML}{721045}
\newcommand{\EFrb}[1]{\textcolor{EFrb}{#1}} % font-lock-regexp-grouping-backslash
\definecolor{Efob}{HTML}{f2f2f2}
\newcommand{\EFob}[1]{\colorbox{Efob}{\efstrut{}#1}} % org-block
\definecolor{EFobb}{HTML}{595959}
\definecolor{Efobb}{HTML}{f2f2f2}
\newcommand{\EFobb}[1]{\colorbox{Efobb}{\efstrut{}\textcolor{EFobb}{#1}}} % org-block-begin-line
\definecolor{EFobe}{HTML}{595959}
\definecolor{Efobe}{HTML}{f2f2f2}
\newcommand{\EFobe}[1]{\colorbox{Efobe}{\efstrut{}\textcolor{EFobe}{#1}}} % org-block-end-line
\newcommand{\EFOa}[1]{\textbf{#1}} % outline-1
\definecolor{EFOb}{HTML}{624416}
\newcommand{\EFOb}[1]{\textcolor{EFOb}{\textbf{#1}}} % outline-2
\definecolor{EFOc}{HTML}{193668}
\newcommand{\EFOc}[1]{\textcolor{EFOc}{\textbf{#1}}} % outline-3
\definecolor{EFOd}{HTML}{721045}
\newcommand{\EFOd}[1]{\textcolor{EFOd}{\textbf{#1}}} % outline-4
\definecolor{EFOe}{HTML}{2a5045}
\newcommand{\EFOe}[1]{\textcolor{EFOe}{\textbf{#1}}} % outline-5
\definecolor{EFOf}{HTML}{7f0000}
\newcommand{\EFOf}[1]{\textcolor{EFOf}{\textbf{#1}}} % outline-6
\definecolor{EFOg}{HTML}{3f578f}
\newcommand{\EFOg}[1]{\textcolor{EFOg}{\textbf{#1}}} % outline-7
\definecolor{EFOh}{HTML}{595959}
\newcommand{\EFOh}[1]{\textcolor{EFOh}{\textbf{#1}}} % outline-8
\definecolor{EFhn}{HTML}{0000b0}
\newcommand{\EFhn}[1]{\textcolor{EFhn}{#1}} % highlight-numbers-number
\definecolor{EFhq}{HTML}{531ab6}
\newcommand{\EFhq}[1]{\textcolor{EFhq}{#1}} % highlight-quoted-quote
\definecolor{EFhs}{HTML}{0000b0}
\newcommand{\EFhs}[1]{\textcolor{EFhs}{#1}} % highlight-quoted-symbol
\newcommand{\EFrda}[1]{#1} % rainbow-delimiters-depth-1-face
\definecolor{EFrdb}{HTML}{dd22dd}
\newcommand{\EFrdb}[1]{\textcolor{EFrdb}{#1}} % rainbow-delimiters-depth-2-face
\definecolor{EFrdc}{HTML}{008899}
\newcommand{\EFrdc}[1]{\textcolor{EFrdc}{#1}} % rainbow-delimiters-depth-3-face
\definecolor{EFrdd}{HTML}{972500}
\newcommand{\EFrdd}[1]{\textcolor{EFrdd}{#1}} % rainbow-delimiters-depth-4-face
\definecolor{EFrde}{HTML}{808000}
\newcommand{\EFrde}[1]{\textcolor{EFrde}{#1}} % rainbow-delimiters-depth-5-face
\definecolor{EFrdf}{HTML}{531ab6}
\newcommand{\EFrdf}[1]{\textcolor{EFrdf}{#1}} % rainbow-delimiters-depth-6-face
\definecolor{EFrdg}{HTML}{008900}
\newcommand{\EFrdg}[1]{\textcolor{EFrdg}{#1}} % rainbow-delimiters-depth-7-face
\definecolor{EFrdh}{HTML}{3548cf}
\newcommand{\EFrdh}[1]{\textcolor{EFrdh}{#1}} % rainbow-delimiters-depth-8-face
\definecolor{EFrdi}{HTML}{8f0075}
\newcommand{\EFrdi}[1]{\textcolor{EFrdi}{#1}} % rainbow-delimiters-depth-9-face
\definecolor{EFany}{HTML}{6f5500}
\definecolor{Efany}{HTML}{6f5500}
\newcommand{\EFany}[1]{\colorbox{Efany}{\efstrut{}\textcolor{EFany}{#1}}} % ansi-color-yellow
\definecolor{EFanr}{HTML}{a60000}
\definecolor{Efanr}{HTML}{a60000}
\newcommand{\EFanr}[1]{\colorbox{Efanr}{\efstrut{}\textcolor{EFanr}{#1}}} % ansi-color-red
\definecolor{EFanb}{HTML}{000000}
\definecolor{Efanb}{HTML}{000000}
\newcommand{\EFanb}[1]{\colorbox{Efanb}{\efstrut{}\textcolor{EFanb}{#1}}} % ansi-color-black
\definecolor{EFang}{HTML}{006800}
\definecolor{Efang}{HTML}{006800}
\newcommand{\EFang}[1]{\colorbox{Efang}{\efstrut{}\textcolor{EFang}{#1}}} % ansi-color-green
\definecolor{EFanB}{HTML}{0031a9}
\definecolor{EfanB}{HTML}{0031a9}
\newcommand{\EFanB}[1]{\colorbox{EfanB}{\efstrut{}\textcolor{EFanB}{#1}}} % ansi-color-blue
\definecolor{EFanc}{HTML}{005e8b}
\definecolor{Efanc}{HTML}{005e8b}
\newcommand{\EFanc}[1]{\colorbox{Efanc}{\efstrut{}\textcolor{EFanc}{#1}}} % ansi-color-cyan
\definecolor{EFanw}{HTML}{a6a6a6}
\definecolor{Efanw}{HTML}{a6a6a6}
\newcommand{\EFanw}[1]{\colorbox{Efanw}{\efstrut{}\textcolor{EFanw}{#1}}} % ansi-color-white
\definecolor{EFanm}{HTML}{721045}
\definecolor{Efanm}{HTML}{721045}
\newcommand{\EFanm}[1]{\colorbox{Efanm}{\efstrut{}\textcolor{EFanm}{#1}}} % ansi-color-magenta
\definecolor{EFANy}{HTML}{884900}
\definecolor{EfANy}{HTML}{884900}
\newcommand{\EFANy}[1]{\colorbox{EfANy}{\efstrut{}\textcolor{EFANy}{#1}}} % ansi-color-bright-yellow
\definecolor{EFANr}{HTML}{972500}
\definecolor{EfANr}{HTML}{972500}
\newcommand{\EFANr}[1]{\colorbox{EfANr}{\efstrut{}\textcolor{EFANr}{#1}}} % ansi-color-bright-red
\definecolor{EFANb}{HTML}{595959}
\definecolor{EfANb}{HTML}{595959}
\newcommand{\EFANb}[1]{\colorbox{EfANb}{\efstrut{}\textcolor{EFANb}{#1}}} % ansi-color-bright-black
\definecolor{EFANg}{HTML}{00663f}
\definecolor{EfANg}{HTML}{00663f}
\newcommand{\EFANg}[1]{\colorbox{EfANg}{\efstrut{}\textcolor{EFANg}{#1}}} % ansi-color-bright-green
\definecolor{EFANB}{HTML}{3548cf}
\definecolor{EfANB}{HTML}{3548cf}
\newcommand{\EFANB}[1]{\colorbox{EfANB}{\efstrut{}\textcolor{EFANB}{#1}}} % ansi-color-bright-blue
\definecolor{EFANc}{HTML}{005f5f}
\definecolor{EfANc}{HTML}{005f5f}
\newcommand{\EFANc}[1]{\colorbox{EfANc}{\efstrut{}\textcolor{EFANc}{#1}}} % ansi-color-bright-cyan
\definecolor{EFANw}{HTML}{ffffff}
\definecolor{EfANw}{HTML}{ffffff}
\newcommand{\EFANw}[1]{\colorbox{EfANw}{\efstrut{}\textcolor{EFANw}{#1}}} % ansi-color-bright-white
\definecolor{EFANm}{HTML}{531ab6}
\definecolor{EfANm}{HTML}{531ab6}
\newcommand{\EFANm}[1]{\colorbox{EfANm}{\efstrut{}\textcolor{EFANm}{#1}}} % ansi-color-bright-magenta
\begin{document}

\maketitle
\section*{Treści zadań}
\label{sec:org43b7359}
\subsection*{Zadanie 1}
\label{sec:org17a8b03}
Zaimplementować semafor binarny za pomocą metod wait i notify, użyć go do
synchronizacji programu Wyścig.
\subsection*{Zadanie 2}
\label{sec:orgd99b04c}
Pokazać, ze do implementacji semafora za pomocą metod wait i \texttt{notify} nie
wystarczy instrukcja \texttt{if} tylko potrzeba użyć \texttt{while}. Wyjaśnić teoretycznie
dlaczego i potwierdzić eksperymentem w praktyce.(wskazówka: rozważyć dwie
kolejki: czekająca na wejście do monitora obiektu oraz kolejkę związana z
instrukcją \texttt{wait}, rozważyć kto kiedy jest budzony i kiedy następuje wyścig).
\subsection*{Zadanie 3}
\label{sec:orgcf31ab0}
Zaimplementować semafor licznikowy (ogólny) za pomocą semaforów binarnych. Czy
semafor binarny jest szczególnym przypadkiem semafora ogólnego?
\section*{Zadanie 1}
\label{sec:orgecd68bf}

Implementujemy dwie metody semafora binarnego:

\begin{itemize}
\item \texttt{P()}: próbuje uzyskać dostęp do zasobu. Jeśli
zasób jest zajęty (\texttt{isAvailable} jest \texttt{false}), wątek zwiększa licznik \texttt{waiting} i
czeka (\texttt{wait()}). Gdy zasób staje się dostępny, wątek zmniejsza licznik \texttt{waiting}
i ustawia \texttt{isAvailable} na \texttt{false}.
\item \texttt{V()}: zwalnia zasób. Ustawia \texttt{isAvailable} na \texttt{true}
i powiadamia (\texttt{notify()}) jeden z oczekujących wątków, jeśli taki istnieje
(\texttt{waiting > 0}).
\end{itemize}

\begin{Code}
\begin{Verbatim}
\color{EFD}\EFk{class} \EFt{BinarySemaphore} \EFrda{\{}
    \EFk{private} \EFt{boolean} \EFv{isAvailable} = \EFo{true};
    \EFk{private} \EFt{int} \EFv{waiting} = \EFhn{0};

    \EFk{public} BinarySemaphore\EFrdb{(}\EFrdb{)} \EFrdb{\{}\EFrdb{\}}

    \EFk{public} \EFk{synchronized} \EFt{void} \EFf{P}\EFrdb{(}\EFrdb{)} \EFrdb{\{}
        \EFk{while} \EFrdc{(}\EFnc{!}isAvailable\EFrdc{)} \EFrdc{\{}
            waiting++;
            \EFk{try} \EFrdd{\{}
                wait\EFrda{(}\EFrda{)};
            \EFrdd{\}} \EFk{catch} \EFrdd{(}InterruptedException e\EFrdd{)} \EFrdd{\{}
                e.printStackTrace\EFrda{(}\EFrda{)};
            \EFrdd{\}}
            waiting--;
        \EFrdc{\}}
        isAvailable = \EFo{false};

    \EFrdb{\}}

    \EFk{public} \EFk{synchronized} \EFt{void} \EFf{V}\EFrdb{(}\EFrdb{)} \EFrdb{\{}
        isAvailable = \EFo{true};
        \EFk{if} \EFrdc{(}waiting > \EFhn{0}\EFrdc{)} \EFrdc{\{}
            notify\EFrdd{(}\EFrdd{)};
        \EFrdc{\}}
    \EFrdb{\}}
\EFrda{\}}
\end{Verbatim}
\end{Code}

Aby zsynchronizować program Wyścig, do klasy każdego wątku podajemy semafora przez
konstruktor, następnie w metodzie \texttt{run()} oba wątki, używają semafora w celu
synchronizacji.

Klasa \texttt{IThread}:
\begin{Code}
\begin{Verbatim}
\color{EFD}\EFk{class} \EFt{IThread} \EFk{extends} \EFt{Thread} \EFrda{\{}
    \EFk{private} \EFk{final} \EFt{Counter} \EFv{\_cnt};
    \EFk{private} \EFk{final} \EFt{BinarySemaphore} \EFv{\_sem};

    \EFk{public} IThread\EFrdb{(}\EFt{Counter} \EFv{c}, BinarySemaphore s\EFrdb{)} \EFrdb{\{}
        \_cnt = c;
        \_sem = s;
    \EFrdb{\}}
    \EFk{public} \EFt{void} \EFf{run}\EFrdb{(}\EFrdb{)} \EFrdb{\{}
        \EFk{for} \EFrdc{(}\EFt{int} \EFv{i} = \EFhn{0}; i < \EFhn{100000000}; ++i\EFrdc{)} \EFrdc{\{}
            \EFo{\_sem}.P\EFrdd{(}\EFrdd{)};
            \_cnt.inc\EFrdd{(}\EFrdd{)};
            \EFo{\_sem}.V\EFrdd{(}\EFrdd{)};
        \EFrdc{\}}
    \EFrdb{\}}
\EFrda{\}}
\end{Verbatim}
\end{Code}


Klasa \texttt{DThread}:
\begin{Code}
\begin{Verbatim}
\color{EFD}\EFk{class} \EFt{DThread} \EFk{extends} \EFt{Thread} \EFrda{\{}
    \EFk{private} \EFk{final} \EFt{Counter} \EFv{\_cnt};
    \EFk{private} \EFk{final} \EFt{BinarySemaphore} \EFv{\_sem};

    \EFk{public} DThread\EFrdb{(}\EFt{Counter} \EFv{c}, BinarySemaphore s\EFrdb{)} \EFrdb{\{}
        \_cnt = c;
        \_sem = s;
    \EFrdb{\}}

    \EFk{public} \EFt{void} \EFf{run}\EFrdb{(}\EFrdb{)} \EFrdb{\{}
        \EFk{for} \EFrdc{(}\EFt{int} \EFv{i} = \EFhn{0}; i < \EFhn{100000000}; ++i\EFrdc{)} \EFrdc{\{}
            \EFo{\_sem}.P\EFrdd{(}\EFrdd{)};
            \_cnt.dec\EFrdd{(}\EFrdd{)};
            \EFo{\_sem}.V\EFrdd{(}\EFrdd{)};
        \EFrdc{\}}
    \EFrdb{\}}
\EFrda{\}}
\end{Verbatim}
\end{Code}

W metodzie \texttt{main} podajemy semafora do wątków,
uruchamiamy je, czekamy aż się zakończą i sprawdzamy finalny wynik.
\begin{Code}
\begin{Verbatim}
\color{EFD}\EFk{public} \EFk{static} \EFt{void} \EFf{main}\EFrda{(}\EFt{String}\EFrdb{[}\EFrdb{]} \EFv{args}\EFrda{)} \EFrda{\{}
\EFt{Counter} \EFv{cnt} = \EFk{new} \EFt{Counter}\EFrdb{(}\EFhn{0}\EFrdb{)};

\EFt{BinarySemaphore} \EFv{sem} = \EFk{new} \EFt{BinarySemaphore}\EFrdb{(}\EFrdb{)};

\EFt{IThread} \EFv{it} = \EFk{new} \EFt{IThread}\EFrdb{(}cnt, sem\EFrdb{)};
\EFt{DThread} \EFv{dt} = \EFk{new} \EFt{DThread}\EFrdb{(}cnt, sem\EFrdb{)};

it.start\EFrdb{(}\EFrdb{)};
dt.start\EFrdb{(}\EFrdb{)};

\EFk{try} \EFrdb{\{}
        it.join\EFrdc{(}\EFrdc{)};
        dt.join\EFrdc{(}\EFrdc{)};
\EFrdb{\}} \EFk{catch}\EFrdb{(}InterruptedException ie\EFrdb{)} \EFrdb{\{} \EFrdb{\}}

System.out.println\EFrdb{(}\EFs{"value="} + cnt.value\EFrdc{(}\EFrdc{)}\EFrdb{)};
\EFrda{\}}
\end{Verbatim}
\end{Code}

Wynik działania programu:
\begin{tcolorbox}
\begin{Verbatim}
value=0
\end{Verbatim}



\end{tcolorbox}\textbf{Wnioski}: Stworzony przez nas semafor binarny może być użyty w celu synchronizacji
działania wielu wątków, w celu wyeliminowania wyścigu.
\section*{Zadanie 2}
\label{sec:org512295d}
Aby pokazać, że użycie instrukcji \texttt{if} nie wystarczy do implementacji semafora
zademonstrujemy przykład pokazujący wyścig w takiej sytuacji.

Mamy trzy wątki:
\begin{center}
\begin{tabular}{ll}
nazwa & stan\\
\hline
\(a\) & jest uśpiony w \texttt{P()} (wait set)\\
\(b\) & będzie chciał wywołać \texttt{P()}\\
\(c\) & obecnie wywołuje metodę \texttt{V()} (posiada monitor)\\
\end{tabular}
\end{center}

Rozważmy następujący przeplot:
\begin{enumerate}
\item \(b\) chce wywołać metodę \texttt{P()}, ale obecnie \(c\) posiada monitor, wątek \(b\) trafia więc
do kolejki monitora(entry set).
\item \(c\) w metodzie \texttt{V()} wywołuje \texttt{notify}, co sprawia, że jedyny wątek w wait set
czyli wątek \(a\) jest oznaczony do wykonania. Wątek kończy metodę i zwalnia monitor.
\item \(b\) jest w entry set, \(a\) w wait set. Zająć monitor może którykolwiek z nich.
\(b\) zajmuje monitor, wywołuje metodę \texttt{P()}, ustawiając \texttt{isAvailable=false} i
ostatecznie zwalnia monitor.
\item Przez oznaczenie wątku \(a\) w kroku 2. wątek zajmuje monitor, ale przez
brak pętli \texttt{while}, wątek ten ustawia \texttt{isAvailable=false} i zaczyna wykonywać swoją sekcję krytyczną.
\end{enumerate}

Jak widać mamy tutaj do czynienia z sytuacją absolutnie niedopuszczalną, bo
wątki \(b\) i \(a\) myślą, że mają ekskluzywny dostęp do danego zasobu.  Gdybyśmy
zastąpili \texttt{if} pętlą \texttt{while} wątek \(a\) po wybudzeniu, ponownie sprawdziłby warunek i
znów trafił do wait set.

Aby pokazać, że wymagana jest w implementacji pętla \texttt{while}, w metodzie \texttt{P()}
zamieniamy instrukcję \texttt{while} na \texttt{if}.
Następnie w metodzie \texttt{main} tworzymy cztery wątki w celu zwiększenia szans wyścigu:
\begin{Code}
\begin{Verbatim}
\color{EFD}\EFk{public} \EFk{static} \EFt{void} \EFf{main}\EFrda{(}\EFt{String}\EFrdb{[}\EFrdb{]} \EFv{args}\EFrda{)} \EFrda{\{}
    \EFt{Counter} \EFv{cnt} = \EFk{new} \EFt{Counter}\EFrdb{(}\EFhn{0}\EFrdb{)};

    \EFt{BinarySemaphore} \EFv{sem} = \EFk{new} \EFt{BinarySemaphore}\EFrdb{(}\EFrdb{)};

    \EFt{List}\EFrdb{<}\EFt{Thread}\EFrdb{>} \EFv{threads} = List.of\EFrdb{(}
        \EFk{new} \EFt{IThread}\EFrdc{(}cnt, sem\EFrdc{)},
        \EFk{new} \EFt{DThread}\EFrdc{(}cnt, sem\EFrdc{)},
        \EFk{new} \EFt{IThread}\EFrdc{(}cnt, sem\EFrdc{)},
        \EFk{new} \EFt{DThread}\EFrdc{(}cnt, sem\EFrdc{)}
    \EFrdb{)};


    threads.forEach\EFrdb{(}Thread::start\EFrdb{)};

    \EFk{try} \EFrdb{\{}
        \EFk{for} \EFrdc{(}\EFt{Thread} \EFv{t} : threads\EFrdc{)} \EFrdc{\{}
            t.join\EFrdd{(}\EFrdd{)};
        \EFrdc{\}}
    \EFrdb{\}} \EFk{catch}\EFrdb{(}InterruptedException ie\EFrdb{)} \EFrdb{\{} \EFrdb{\}}

    System.out.println\EFrdb{(}\EFs{"value="} + cnt.value\EFrdc{(}\EFrdc{)}\EFrdb{)};
\EFrda{\}}
\end{Verbatim}
\end{Code}

Wynik programu:
\begin{tcolorbox}
\begin{Verbatim}
value=-689
\end{Verbatim}


\end{tcolorbox}Jak widzimy wynik odbiega od poprawnego zera.

\textbf{Wnioski}: Przy implementacji niskopoziomowych narzędzi synchronizacji musi być
bardzo uważni. Wymóg zastosowania pętli \texttt{while} na pierwszy rzut oka wcale nie musi być
oczywisty. Jego konieczność wynika z zasady działania monitora w języku Java.
Implementując zatem takie narzędzia musimy być niezwykle pewni tego na
jakich innych narzędziach je bazujemy.
\section*{Zadanie 3}
\label{sec:org234408c}
Aby zaimplementować semafor licznikowy wykorzystamy zmienną \texttt{counter} oraz dwa semafory:
\begin{itemize}
\item \texttt{mutex} - chroni zmienną \texttt{counter}
\item \texttt{gate} - ten semafor jest podniesiony gdy \texttt{counter > 0}, czyli wtedy
gdy chociaż jeden zasób jest nadal dostępny.
\end{itemize}

\begin{Code}
\begin{Verbatim}
\color{EFD}\EFk{class} \EFt{CountingSemaphore} \EFrda{\{}
    \EFk{private} \EFk{final} \EFt{BinarySemaphore} \EFv{gate};
    \EFk{private} \EFk{final} \EFt{BinarySemaphore} \EFv{mutex};
    \EFk{private} \EFt{int} \EFv{counter};

    \EFk{public} CountingSemaphore\EFrdb{(}\EFt{int} \EFv{n}\EFrdb{)} \EFrdb{\{}
        mutex = \EFk{new} \EFt{BinarySemaphore}\EFrdc{(}\EFrdc{)};
        gate = \EFk{new} \EFt{BinarySemaphore}\EFrdc{(}\EFrdc{)};
        counter = n;
        \EFk{if} \EFrdc{(}counter == \EFhn{0}\EFrdc{)} \EFrdc{\{}
            \EFo{gate}.P\EFrdd{(}\EFrdd{)};
        \EFrdc{\}}
    \EFrdb{\}}

    \EFk{public} \EFt{void} \EFf{P}\EFrdb{(}\EFrdb{)} \EFrdb{\{}
        \EFo{gate}.P\EFrdc{(}\EFrdc{)};
        \EFo{mutex}.P\EFrdc{(}\EFrdc{)};
        counter--;
        \EFk{if} \EFrdc{(}counter > \EFhn{0}\EFrdc{)} \EFrdc{\{}
            \EFo{gate}.V\EFrdd{(}\EFrdd{)};
        \EFrdc{\}}
        \EFo{mutex}.V\EFrdc{(}\EFrdc{)};
    \EFrdb{\}}

    \EFk{public} \EFt{void} \EFf{V}\EFrdb{(}\EFrdb{)} \EFrdb{\{}
        \EFo{mutex}.P\EFrdc{(}\EFrdc{)};
        counter++;
        \EFk{if} \EFrdc{(}counter >= \EFhn{1}\EFrdc{)} \EFrdc{\{}
            \EFo{gate}.V\EFrdd{(}\EFrdd{)};
        \EFrdc{\}}
        \EFo{mutex}.V\EFrdc{(}\EFrdc{)};
    \EFrdb{\}}
\EFrda{\}}
\end{Verbatim}
\end{Code}

Aby zademonstrować działanie tej klasy zasymulujemy jedno z częstych zastosowań
dla takiego semafora. Ograniczymy ilość otwartych gniazd przez nasz program, z
racji tego, że ta liczba w systemie ma najczęściej swoją górną granicę.
Przy naiwnym programie typu web-scraper, bez kontroli ilości otwartych gniazd
moglibyśmy przekroczyć ten limit co skutkowałoby wyjątkami, system odmówił by
utworzenia nowego gniazda.

Kod wątku który symuluje otwarcie gniazda:
\begin{Code}
\begin{Verbatim}
\color{EFD}\EFk{class} \EFt{OpenNetworkSocketThread} \EFk{extends} \EFt{Thread} \EFrda{\{}
    \EFk{private} \EFk{final} \EFt{CountingSemaphore} \EFv{\_sem};
    \EFk{private} \EFk{final} \EFt{int} \EFv{id};

    \EFk{public} OpenNetworkSocketThread\EFrdb{(}\EFt{int} \EFv{id}, CountingSemaphore s\EFrdb{)} \EFrdb{\{}
        \EFk{this}.id = id;
        \_sem = s;
    \EFrdb{\}}

    \EFk{public} \EFt{void} \EFf{run}\EFrdb{(}\EFrdb{)} \EFrdb{\{}
        \EFo{\_sem}.P\EFrdc{(}\EFrdc{)};
        System.out.println\EFrdc{(}\EFs{"Opened network socket in "} + id\EFrdc{)};
        \EFk{try} \EFrdc{\{}
            Thread.sleep\EFrdd{(}ThreadLocalRandom.current\EFrda{(}\EFrda{)}.nextInt\EFrda{(}\EFhn{100}, \EFhn{300}\EFrda{)}\EFrdd{)};
        \EFrdc{\}} \EFk{catch} \EFrdc{(}InterruptedException e\EFrdc{)} \EFrdc{\{}
            e.printStackTrace\EFrdd{(}\EFrdd{)};
        \EFrdc{\}}
        System.out.println\EFrdc{(}\EFs{"Closing network socket in "} + id\EFrdc{)};
        \EFo{\_sem}.V\EFrdc{(}\EFrdc{)};
    \EFrdb{\}}
\EFrda{\}}
\end{Verbatim}
\end{Code}

Metoda \texttt{main}.
\begin{Code}
\begin{Verbatim}
\color{EFD}\EFk{public} \EFk{static} \EFt{void} \EFf{main}\EFrda{(}\EFt{String}\EFrdb{[}\EFrdb{]} \EFv{args}\EFrda{)} \EFrda{\{}
    \EFt{var} \EFv{sem} = \EFk{new} \EFt{CountingSemaphore}\EFrdb{(}\EFhn{3}\EFrdb{)};
    \EFt{List}\EFrdb{<}\EFt{OpenNetworkSocketThread}\EFrdb{>} \EFv{threads} =
        IntStream.range\EFrdb{(}\EFhn{0}, \EFhn{10}\EFrdb{)}
            .mapToObj\EFrdb{(}i -> \EFk{new} \EFt{OpenNetworkSocketThread}\EFrdc{(}i, sem\EFrdc{)}\EFrdb{)}
            .toList\EFrdb{(}\EFrdb{)};

    theads.forEach\EFrdb{(}Thread::start\EFrdb{)};
    threads.forEach\EFrdb{(}t -> \EFrdc{\{}
        \EFk{try} \EFrdd{\{}
            t.join\EFrda{(}\EFrda{)};
        \EFrdd{\}} \EFk{catch} \EFrdd{(}InterruptedException e\EFrdd{)} \EFrdd{\{}
            e.printStackTrace\EFrda{(}\EFrda{)};
        \EFrdd{\}}
    \EFrdc{\}}\EFrdb{)};
\EFrda{\}}
\end{Verbatim}
\end{Code}

Wynik programu:
\begin{tcolorbox}
\begin{Verbatim}
Opened network socket in 1
Opened network socket in 0
Opened network socket in 2
Closing network socket in 1
Opened network socket in 3
Closing network socket in 2
Opened network socket in 4
Closing network socket in 0
Opened network socket in 6
Closing network socket in 3
Opened network socket in 5
Closing network socket in 4
Opened network socket in 7
Closing network socket in 6
Opened network socket in 8
Closing network socket in 5
Opened network socket in 9
Closing network socket in 7
Closing network socket in 8
Closing network socket in 9
\end{Verbatim}


\end{tcolorbox}Jak widać stworzony przez nas semafor poprawnie umożliwia utworzenie jedynie trzech gniazd w
danym momencie.

Semafor binarny jest szczególnym przypadkiem semafora licznikowego. Semafor
licznikowy zachowuje się tak samo jak semafor binarny gdy liczba zasobów jest
równa jeden. Nie oznacza to jednak, że zawsze powinniśmy korzystać z semaforów
licznikowych skoro są ogólniejsze od semaforów binarnych. Semafor binarny
ma prostszą implementację(licznikowy składa się właśnie z dwóch takich semaforów),
przez to jeżeli potrzebujemy zwykłego mutex'a to powinniśmy właśnie użyć semafora binarnego.
\section*{Bibliografia}
\label{sec:org7dbccd1}
\begin{itemize}
\item Bill Venners: \emph{Inside the Java Virtual Machine Chapter 20}
\end{itemize}
\end{document}
